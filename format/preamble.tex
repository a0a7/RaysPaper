% This is preamble.tex



% Set default input encoding.
\usepackage[utf8]{inputenc}

%%% If something looks strange, try one of the input encodings below instead.
%\usepackage[latin1]{inputenc} 
%\usepackage[applemac]{inputenc}
%\usepackage[ansinew]{inputenc}

% Set your default font encoding.
\usepackage[T1]{fontenc}

% Set default language priority order.
\usepackage[english]{babel}

% Set default margins.
\usepackage[a4paper,margin=25mm,headsep=5mm,headheight=12pt]{geometry}

% If you need to define macros or include more packages, do all that here.
\frenchspacing
\usepackage{amsmath,amssymb,amsfonts,amsthm}  % Mathematics

\usepackage{hyperref} 
\hypersetup{colorlinks, 
    citecolor=black,
    filecolor=black, 
    linkcolor=black,
    urlcolor=black
      }


\usepackage{setspace}
\usepackage{graphicx}
\usepackage{subcaption}
\usepackage{physics} %A lot of useful short commands for writing math/physics
\usepackage[separate-uncertainty=true]{siunitx} %Use this for writing SI units
\sisetup{
  range-phrase=--,
  detect-all,
  output-decimal-marker={.},
  %round-mode=places, % Use these if you want to force yourself to think about how many decimals you actually have
  %round-precision=0, % 
  range-units=single,
  per-mode=fraction, % change fraction to reciprocal if you want ^{-1}
}
\usepackage{url}
\usepackage{verbatim}
\usepackage[toc,acronym]{glossaries}
\usepackage{booktabs} % For nice tables
\usepackage{fancyhdr} % For header and footer
\usepackage{lipsum}
\usepackage{lastpage} % For page counter
\usepackage{svg} % Makes it possible to use .svg files 
\usepackage{float} % Just [H] for placing figures
\usepackage{listings} % Enables source code listings
\usepackage{comment} % Use \begin{comment}